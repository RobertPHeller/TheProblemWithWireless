%%%%%%%%%%%%%%%%%%%%%%%%%%%%%%%%%%%%%%%%%%%%%%%%%%%%%%%%%%%%%%%%%%%%%%%%%%%%%
%
%  System        : 
%  Module        : 
%  Object Name   : $RCSfile$
%  Revision      : $Revision$
%  Date          : $Date$
%  Author        : $Author$
%  Created By    : Robert Heller
%  Created       : Thu Apr 19 13:49:34 2018
%  Last Modified : <180503.1249>
%
%  Description 
%
%  Notes
%
%  History
% 
%%%%%%%%%%%%%%%%%%%%%%%%%%%%%%%%%%%%%%%%%%%%%%%%%%%%%%%%%%%%%%%%%%%%%%%%%%%%%
%
%    Copyright (C) 2018  Robert Heller D/B/A Deepwoods Software
%			51 Locke Hill Road
%			Wendell, MA 01379-9728
%
%    This program is free software; you can redistribute it and/or modify
%    it under the terms of the GNU General Public License as published by
%    the Free Software Foundation; either version 2 of the License, or
%    (at your option) any later version.
%
%    This program is distributed in the hope that it will be useful,
%    but WITHOUT ANY WARRANTY; without even the implied warranty of
%    MERCHANTABILITY or FITNESS FOR A PARTICULAR PURPOSE.  See the
%    GNU General Public License for more details.
%
%    You should have received a copy of the GNU General Public License
%    along with this program; if not, write to the Free Software
%    Foundation, Inc., 675 Mass Ave, Cambridge, MA 02139, USA.
%
% 
%
%%%%%%%%%%%%%%%%%%%%%%%%%%%%%%%%%%%%%%%%%%%%%%%%%%%%%%%%%%%%%%%%%%%%%%%%%%%%%

\documentclass[12pt]{article}
\usepackage{times}
\usepackage{url}
\usepackage{graphicx}
\usepackage{mathptm}
\usepackage{makeidx}
\usepackage[pdftex,pagebackref=true]{hyperref}
\hypersetup{%
  colorlinks=true,%
  linkcolor=blue,%
  citecolor=blue,%
  unicode%
}
\title{The Problem With Wireless}
\author{Robert Heller}
\date{\today}
\begin{document}

\maketitle

\tableofcontents
%\listoffigures
%\listoftables


\section{Introduction}

















This paper is my attempt to present the very real problems involved in
providing wireless telecom services in Western Mass. There is a lot of
misunderstanding about what wireless can and cannot do. There is the
misconception that wireless is a ``cheap'' and ``easy'' solution. It is just
not that simple. Part of this misconception is that wireless is seen as almost
``magical''. To quote Dr. Aurthur C. Clark: ``Any sufficiently advanced
technology is indistinguishable from magic.'' To many people wireless
technology has much of the look and feel of magic, since it is hard for people
grasp communication traveling through the air with no apparent ``connection''.
Any kind of wired communication is easier for people to wrap their heads
around ­ not just traditional copper-based, but also fiber-optic or even the
use of a string and a pair of soup cans. In this paper I will try to demystify
the technology of wireless telecom services and the very real and down to
earth limitations. I will also briefly cover some of the economic issues that
arise for telecom services.


\section{Centimeter and Millimeter Radio Wave Propagation Issues in Western Mass}

The terrain of Western Mass presents some specific challenges to wireless
communication. This area is heavily forested and is also very hilly with many
small mountains and ridges, most in fact covered with trees. Wireless Internet
and Cellular Telephone services are using (or will be using) radio frequencies
in the Centimeter and Millimeter wavelengths. These wavelengths are best used
in line-of-sight situations (a problem in hilly and mountainous areas, where
mountains and hills can completely block the signal) and these signals can be
highly attenuated by foliage(trees)\cite{Effect.IJCA.0975-8887}. This means
that to provide full coverage, additional access points or cell towers need to
placed to ``fill in around'' hills and mountains and to simply provide
additional signal sources to account for signal loss due to foliage
attenuation. And often even with additional access points or cell towers,
there will be holes in the coverage. It can be difficult or even impossible to
provide coverage in some areas.

It is important to understand that these issues with centimeter and millimeter
radio wave propagation are related to the underlying \textbf{physics} of radio
wave propagation are not subject to any sort of \textbf{technological}
``fix''. In fact, as frequencies go higher (such as with the soon to be
developed 5G cellular bands that use millimeter wavelengths) the problems of
foliage and blockage get even worse. The main advances in wireless technology
relate to how information is encoded on the carrier wave, including how the
radio waves are modulated and how data is compressed and encrypted. More data
means more bandwidth, and so there is a move to higher carrier frequencies to
accommodate that bandwidth.

There is not going to be some technological advance to make the radio waves at
these wavelengths ``bend'' around hills or somehow push their way through the
trees better. Radio waves are not really magic, even if they seem magical in
some sense. In many important ways, they are like light waves. You can imagine
each cell tower is a lighthouse in an otherwise dark nighttime world. Some
light makes it through the leaves of the forest but not as much as if you are
in the open where you can see the tower. The light may reflect off some
buildings or other objects, but it cannot bend in normal circumstances. Radio
waves are just as much subject to the Laws of Physics as anything else.

\section{The Economic Issues of Providing Service in Western Mass}

A telecom provider can only provide a service if they have an infrastructure.
Because of the way the terrain of Western Mass impacts radio wave propagation,
rather a lot of infrastructure is needed, both towers / access points and also
the fiber optic backbone needed to connect data between network and the towers
/ access points ­ yes, it's not really possible to have a ``pure'' wireless
option, since the towers and access points need to be connected together and
to the public Internet and wired phone network. This all requires a large
capital investment and a lot of ongoing maintenance.

The capital investment and ongoing maintenance are recovered from fees
collected from the users (subscribers) of the service(s). Because there is a
limit on how much a telecom provider can reasonably collect from its
subscribers, there is a corresponding limit on how much revenue a telecom
provider can expect to collect to pay for and operate its infrastructure.
Since much of rural Western Mass is sparsely populated, this means that the
revenue is strictly limited, and might not be enough to cover the costs for an
extensive infrastructure. At present Cellular service coverage is pretty much
limited to either densely populated areas (like the larger towns) or along
well traveled roads (like numbered state and federal highways). These economic
limitations mean wireless infrastructure may not be any more sustainable than
wired infrastructure ­ a wireless or hybrid system might not be more cost
effective than a wired system. Because wireless is actually a faster moving
technology and is also more ``fragile'', it could end up being more expensive
to maintain, even if the capital cost might be lower.
    

\section{Conclusions}

I hope I have presented some information that eliminates some of the mystery 
and misunderstanding of wireless technology and presents some of the problems 
implementing a wireless telecom solution.


\appendix
\clearpage
\addcontentsline{toc}{section}{References}
\bibliography{TheProblemWithWireless.bib}
\bibliographystyle{plain}
\clearpage
\addcontentsline{toc}{section}{Acknowledgments}
\section*{Acknowledgments}

I would like to thank Thomas Williams, a resident of the neighboring town of
Shutesbury who is also a Ham Radio operator, who read a draft of this paper
and provided some useful feedback. Thanks Tom!

\end{document}
